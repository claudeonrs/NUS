\documentclass{article}
\usepackage[a4paper, left=15mm, top=20mm, right=15mm,bottom=20mm]{geometry}
\usepackage{amsmath, amssymb, amsfonts}
\usepackage{fancyhdr}
\usepackage{graphicx}
\graphicspath{ {./images/} }
\usepackage{float}
\usepackage{hyperref}
\usepackage{lscape}
\usepackage{arev}

\pagestyle{fancy}
\fancyhf{}
\lhead{EC1101E}
\rhead{claudeonrs}
\rfoot{\thepage}
\usepackage{amsmath, amssymb, amsfonts, listings}
\usepackage{xcolor}
\usepackage{enumitem}
\setlist[itemize]{noitemsep, topsep=0pt}
\setlist[enumerate]{noitemsep, topsep=0pt}


%New colors defined below
\definecolor{codegreen}{rgb}{0,0.6,0.4}
\definecolor{codegray}{rgb}{0.5,0.5,0.5}
\definecolor{codepurple}{rgb}{0.58,0,0.82}
\definecolor{backcolour}{rgb}{0.95,0.95,0.92}
\definecolor{commentgreen}{rgb}{0.4,0.8,0.6}
%Code listing style named "mystyle"
\lstdefinestyle{mystyle}{
  backgroundcolor=\color{backcolour},   
  commentstyle=\color{red},
  keywordstyle=\color{blue},
  numberstyle=\tiny\color{codegray},
  stringstyle=\color{codegreen},
  basicstyle=\ttfamily,
  breakatwhitespace=false,         
  breaklines=true,                 
  captionpos=b,                    
  keepspaces=true,                 
  numbers=left,                    
  numbersep=5pt,                  
  showspaces=false,                
  showstringspaces=false,
  showtabs=false,                  
  tabsize=2
}

%"mystyle" code listing set
\lstset{style=mystyle}

\title{No Title}
\author{Claudeon R Susanto}
\date{}
\usepackage[T1]{fontenc}
\usepackage[utf8]{inputenc}
\usepackage[english]{babel}
\usepackage{lmodern}

\renewcommand{\familydefault}{\sfdefault}   % Supprime le serif (dyslexie)
\usepackage[font=sf, labelfont={sf}]{caption}
\usepackage{multicol}
\usepackage{makecell}
\renewcommand\theadalign{bc}
\renewcommand\theadfont{\bfseries}
\renewcommand\theadgape{\Gape[4pt]}
\renewcommand\cellgape{\Gape[4pt]}

\renewcommand\thesubsection{\thesection.\arabic{subsection}}
\setlength{\columnseprule}{1pt}




% own commands
\newcommand{\eg}[0]{\textit{e.g. }}
\newcommand{\ie}[0]{\textit{i.e. }}
\newcommand{\impt}[0]{\textcolor{red}{\textbf{[IMPT] }}}


\begin{document}
%\maketitle
\fontfamily{lmss}\selectfont
\begin{multicols}{2}
\section{Introduction to Economic Analysis}
\subsection{Scarcity}
\textbf{Scarce}: Quantity of resources lower than demand, hence insufficient to satisfv needs and wants\\
\textbf{Resources}: CELL (Capital - physical and human capital, Entrepreneurship, Land, Labour)\\


\textbf{What is Economics?}: study of \underline{choice} under \underline{scarcity}
\begin{itemize}
	\item How \underline{people} decide how much to work, what to buy, how much to save, how to invest, etc. given budget and costs
	\item How \underline{firms} decide how much to produce, how many workers to hire, etc. given available budget and costs
	\item How \underline{society} decides how to allocate its resources among national defense, health care, education, scientific research, social safety nets, etc.\\
\end{itemize}

\textbf{Opportunity cost} of any choice: whatever must be given up when we make that choice
\begin{itemize}
	\item \underline{Explicit cost}: monetary sacrifice
	\item \underline{Implicit cost}: non-monetary e.g. time
	\item \textcolor{red}{\textbf{[IMPT]}} when the alternatives to a choice are mutually exclusive, the implicit cost of the choice is the value of the next best alternative
\end{itemize}

\subsection{Five core principles}
\begin{enumerate}
\item \textbf{Scarcity implies trade-offs}
\begin{itemize}
	\item We have unlimited wants and limited resources
	\item Hence having more of one good thing usually means having less of another.
\end{itemize}
\item \textbf{Bargaining strength comes through scarcity}
\begin{itemize}
	\item Scarce resources command high prices
\end{itemize}
\item \textbf{Compare costs and benefits}
\begin{itemize}
	\item An action should be taken if, and only if, the benefit is at least as great as the cost.
\end{itemize}
\item \textbf{People respond to changes in costs and benefits}
\begin{itemize}
	\item The likelihood of taking an action rises as the benefit rises, and falls as the cost rises.
\end{itemize}
\item \textbf{Focus on your comparative advantage}
\begin{itemize}
	\item  Everyone gains when each individual (or each country) concentrates on the activities in which her opportunity cost is lowest.
\end{itemize}
\end{enumerate}

\subsection{Types of economics}
\textbf{Microeconomics}: derived from \textit{Mikros} or \textit{small}
\begin{itemize}
	\item The study of how households and firms \underline{make decisions} and how they \underline{interact} in markets
\end{itemize}
\textbf{Microeconomics}: derived from \textit{Makros} or \textit{large}
\begin{itemize}
	\item The study of economy-wide phenomena e.g. inflation, unemployment, and econ growth
\end{itemize}
\textbf{Positive Economics}: \textit{describe} the world as it is
\begin{itemize}
	\item Addresses "What is?" question using \underline{tools of economics}, without any \underline{value judgment}
	\item Positive \textbf{statements}: can be confirmed or refuted by examining evidence
	\item Positive \textbf{disagreements}: due to differences in scientific judgments
\end{itemize}
\textbf{Normative Economics}: \textit{prescribe} how the world should be
\begin{itemize}
	\item Addresses "What should be?" question which require \underline{value judgment}
	\item Every normative analysis is based on underlying positive analysis
	\item Normative \textbf{statements}: cannot be confirmed or refuted
	\item Normative \textbf{disagreements}: due to differences in values
\end{itemize}

\subsection{Production Possibility Frontier (PPF)}

\textbf{Model}: A simplification of a more complicated reality
\begin{itemize}
	\item \textit{Simplifying} assumptions: do not affect important conclusions
	\item \textit{Critical} assumptions: affect important conclusions
\end{itemize}
\textbf{Definition}: A graph that shows all combinations of two goods that can be produced given the available resources and technology
\begin{itemize}
	\item Points on the PPF: possible and efficient
	\item Points under the PPF: possible but not efficient
	\item Points above the PPF: not possible
\end{itemize}
\textbf{Moving along} a PPF 
\begin{itemize}
	\item Involves \underline{shifting resources} from the production of one good to the production of the other good 
	\item Because resources are limited and hence sacrifice has to be made
	\item \textbf{Slope} of PPF $=$ \textbf{Opportunity cost} of good $x$ in terms of good $y$ 
	
\end{itemize}
\textbf{Shifting} of PPF
\begin{itemize}
	\item Due to \underline{additional resources} or \underline{improvement} in technology
	\item The economy can produce more of good $x$ or good $y$ or any combination in between
\end{itemize}
\textbf{Shapes} of PPF
\begin{itemize}
	\item Straight line: opp. cost is constant
	\item Concave: the \underline{opp. cost} of a good \underline{rises} as the economy produces more of the good
	\begin{itemize}
		\item When different resources are suited for different uses
		\item Different resources have different opp. costs of producing one good in terms of the other good (e.g. different workers have different skills)
		\item Explanation:
		\begin{itemize}
			\item Initially, most workers including those who are better at producing good B are producing good A $\rightarrow$ to get more good B, we can shift workers who are more efficient in producing B from the production of A to B $\rightarrow$ hence we don't need to give up so many of good A
			\item However, producing more of good B would require shifting workers who are more efficient in A than B $\rightarrow$ hence there would be a huge drop in output of A $\rightarrow$ higher opp. cost
		\end{itemize}
	\end{itemize}
\end{itemize}


\subsection{Gains from Trade}
\textbf{Absolute advantage}: the ability to produce a good using \underline{fewer inputs} than another producer
\begin{itemize}
	\item Producer A can produce the same amount of good $x$ with fewer inputs as compared to producer B
	\item \impt Two countries can gain from trade when each specializes in the good it produces at \underline{lowest cost}
\end{itemize}
\textbf{Comparative advantage}: the ability to produce good at a lower opportunity cost than another producer
\begin{itemize}
	\item Producer A can produce the same amount of good $x$ by giving up fewer of good $y$ as compared to producer B
	\item \impt Absolute advantage is not necessary for comparative advantage
	\item Gains from trade arise from comparative advantage (\underline{differences in opp. costs})
	\item When each country specializes in the good in which it has a comparative advantage, 
	\begin{itemize}
		\item total production in all countries is higher,
		\item the world's economic pie is bigger,
		\item and all countries can gain from trade.
	\end{itemize}
\end{itemize}
Note that there are different possibilities for CA/AA
\begin{itemize}
	\item AA possibilities
	\begin{itemize}
		\item A has AA in both goods
		\item A has AA in good X but B has AA in good Y
		\item Neither has AA in either good
	\end{itemize}
	\item CA possibilities
	\begin{itemize}
		\item A has CA in both goods
		\item A has CA in good X but B has CA in good Y
		\item Neither has CA in either good
	\end{itemize}
\end{itemize}



\end{multicols}






\end{document}