\documentclass{article}
\usepackage[a4paper, left=15mm, top=20mm, right=15mm,bottom=20mm]{geometry}
\usepackage{amsmath, amssymb, amsfonts}
\usepackage{fancyhdr}
\usepackage{graphicx}
\graphicspath{ {./images/} }
\usepackage{float}
\usepackage{hyperref}
\usepackage{lscape}

\pagestyle{fancy}
\fancyhf{}
\lhead{DSA2101}
\rhead{claudeonrs}
\rfoot{\thepage}
\usepackage{amsmath, amssymb, amsfonts, listings}
\usepackage{xcolor}
\usepackage{enumitem}
\setlist[itemize]{noitemsep, topsep=0pt}
\setlist[enumerate]{noitemsep, topsep=0pt}
\setlist[description]{noitemsep, topsep=0pt}


%New colors defined below
\definecolor{codegreen}{rgb}{0,0.6,0.4}
\definecolor{codegray}{rgb}{0.5,0.5,0.5}
\definecolor{codepurple}{rgb}{0.58,0,0.82}
\definecolor{backcolour}{rgb}{0.95,0.95,0.92}
\definecolor{commentgreen}{rgb}{0.4,0.8,0.6}
%Code listing style named "mystyle"
\lstdefinestyle{mystyle}{
  backgroundcolor=\color{backcolour},   
  commentstyle=\color{red},
  keywordstyle=\color{blue},
  numberstyle=\tiny\color{codegray},
  stringstyle=\color{codegreen},
  basicstyle=\ttfamily,
  breakatwhitespace=false,         
  breaklines=true,                 
  captionpos=b,                    
  keepspaces=true,                 
  numbers=left,                    
  numbersep=5pt,                  
  showspaces=false,                
  showstringspaces=false,
  showtabs=false,                  
  tabsize=2
}

%"mystyle" code listing set
\lstset{style=mystyle}

\title{No Title}
\author{Claudeon R Susanto}
\date{}
\usepackage[T1]{fontenc}
\usepackage[utf8]{inputenc}
\usepackage[english]{babel}
\usepackage{lmodern}

\renewcommand{\familydefault}{\sfdefault}   % Supprime le serif (dyslexie)
\usepackage[font=sf, labelfont={sf}]{caption}
\usepackage{multicol}
\usepackage{makecell}
\renewcommand\theadalign{bc}
\renewcommand\theadfont{\bfseries}
\renewcommand\theadgape{\Gape[4pt]}
\renewcommand\cellgape{\Gape[4pt]}



% own commands
\newcommand{\eg}[0]{\textit{e.g. }}
\newcommand{\ie}[0]{\textit{i.e. }}
\newcommand{\impt}[0]{\textcolor{red}{\textbf{[IMPT] }}}


\renewcommand\thesubsection{\thesection.\arabic{subsection}}
\setlength{\columnseprule}{1pt}
\begin{document}
%\maketitle
\fontfamily{lmss}\selectfont
\begin{multicols}{2}
\section{R Programming}
\subsection*{List}
\begin{itemize}
	\item \texttt{[[idx]]}: get element in a list
	\item \texttt{str(ls)}: get \textbf{str}ucture of a list (similar to summary)
	\item \texttt{saveRDS} and \texttt{loadRDS}
\end{itemize}
\subsection*{Recycling Rule}
\begin{itemize}
	\item shorter vectors are recycled until they match the length of the longest vector
	\item the length of the longest vector must be a multiple of the shorter vector in arithmetic operations!
\end{itemize}
\subsection*{Useful functions}
\begin{itemize}
	\item \texttt{sample(x, size, replace, prob)}
	\begin{itemize}
		\item \texttt{size}: length of output vector
		\item \texttt{replace}: if \texttt{TRUE}, then sampling is with replacement
		\item \texttt{prob}: a vector of probability weights
	\end{itemize}
	\item \texttt{rep(x, times, length.out)}
	\item \texttt{table()}
	\item \texttt{args(func)}: list the arguments of a function
	\item \texttt{seq(from, to, by, length)}
	\item \texttt{paste(v1, v2, sep)}: concatenate vectors after converting them to characters
	\begin{itemize}
		\item \texttt{sep}: separator between elements of \texttt{v1} and \texttt{v2}
		\item The recycling rule applies when \texttt{length(v1) != length(v2)}
	\end{itemize}
	\item \texttt{apply} function family: apply function to each row (1) or column (2) 
	\begin{itemize}
		\item \texttt{apply(X, margin, func, ...)}
		\begin{itemize}
			\item Note that \texttt{X} must be a matrix in \texttt{apply}
		\end{itemize}
		\item \texttt{sapply} returns a vector or a matrix
		\item \texttt{lapply} returns a list, useful when the output of the function may not be all of the same length/type
	\end{itemize}
\end{itemize}
\subsection*{Function debugging}
\begin{itemize}
	\item \texttt{cat("...")}: used to print statements
	\item \texttt{browser()}: debugging with breakpoint
\end{itemize}
\subsection*{Important classes}
\subsubsection*{Strings}
\begin{itemize}
	\item Start by importing \texttt{tidyverse} and \texttt{stringr}
	\item Library functions
	\begin{itemize}
		\item \texttt{str\_length}: returns vector of string lengths
		\item \texttt{str\_c(..., sep)}: concatenate strings with optional separator
		\item \texttt{str\_sub(string, start, end)}: returns vector of substrings
	\end{itemize}
	\item Regular expressions (\texttt{str\_view()} to test out regex), \href{https://stringr.tidyverse.org/articles/regular-expressions.html}{\textit{\underline{Tidyverse Article}}}
	\begin{itemize}
		\item to match an \textbf{a} at the beginning of a string
		\begin{verbatim}
			str_view(x, "^a")
		\end{verbatim}
		\item to match an \textbf{a} at the end of a string
		\begin{verbatim}
			str_view(x, "a$")
		\end{verbatim}
		\item to match an \textbf{a} or \textbf{e} at the end of a string
		\begin{verbatim}
			str_view(x, "[ae]$")
		\end{verbatim}
		\item to match a string of 3 chars with \textbf{a} in the middle
		\begin{verbatim}
			str_view(x, ".a.")
		\end{verbatim}
	\end{itemize}
	\item \texttt{str\_detect(vec, regex)}: returns a boolean vector
	\item \texttt{str\_extract(vec, regex)}: returns a vector of strings, particularly helpful for \texttt{".a."} regex
\end{itemize}
\subsubsection*{Factors}
\texttt{factor(vec, levels=c(...))}: convert \texttt{vec} to factors with fixes levels

\subsubsection*{Date}
\begin{itemize}
	\item \texttt{as.Date(x, format)}: convert string x to \texttt{Date} object \\ e.g. \texttt{as.Date("2014/02/22", "\%Y/\%m/\%d")}
	\item \texttt{months(d)}: what month of the year is the date in?
	\item \texttt{weekdays(d)}: what day of the week is the date on?
	\item \texttt{Sys.Date()}
	\item \texttt{cut(x, breaks, labels)}: usually used to group dates that fall into a month/week/quarter
	\begin{itemize}
		\item \texttt{breaks}: numeric vector/string (\texttt{"month", "week"})
		\item \texttt{labels}: if \texttt{TRUE}, return a label vector
	\end{itemize}
	
\end{itemize}

\subsection*{Basic Plotting}
\subsubsection*{\texttt{plot()}}
\begin{itemize}
	\item \texttt{pch}: abbr. for plotting character
	\item \texttt{col}: use string or \texttt{rgb(..., alpha=?)} to change colour
	\item \texttt{cex}: abbr. for character expansion
	\item use \texttt{points()} or \texttt{lines()} to add more stuff to an existing plot
	
\end{itemize}
\subsubsection*{\texttt{barplot()}}

\section{Importing Data}
\end{multicols}






\end{document}