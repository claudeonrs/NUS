\documentclass{article}
\usepackage[a4paper, left=15mm, top=20mm, right=15mm,bottom=20mm]{geometry}
\usepackage{amsmath, amssymb, amsfonts}
\usepackage{fancyhdr}
\usepackage{graphicx}
\graphicspath{ {./images/} }
\usepackage{float}
\usepackage{hyperref}
\usepackage{lscape}
\usepackage{mathtools}

\pagestyle{fancy}
\fancyhf{}
\lhead{DSA2101}
\rhead{claudeonrs}
\rfoot{\thepage}
\usepackage{amsmath, amssymb, amsfonts, listings}
\usepackage{xcolor}
\usepackage{enumitem}
\setlist[itemize]{noitemsep, topsep=0pt}
\setlist[enumerate]{noitemsep, topsep=0pt}
\setlist[description]{noitemsep, topsep=0pt}


%New colors defined below
\definecolor{codegreen}{rgb}{0,0.6,0.4}
\definecolor{codegray}{rgb}{0.5,0.5,0.5}
\definecolor{codepurple}{rgb}{0.58,0,0.82}
\definecolor{backcolour}{rgb}{0.95,0.95,0.92}
\definecolor{commentgreen}{rgb}{0.4,0.8,0.6}
%Code listing style named "mystyle"
\lstdefinestyle{mystyle}{
  backgroundcolor=\color{backcolour},
  commentstyle=\color{red},
  keywordstyle=\color{blue},
  numberstyle=\tiny\color{codegray},
  stringstyle=\color{codegreen},
  basicstyle=\ttfamily,
  breakatwhitespace=false,
  breaklines=true,
  captionpos=b,
  keepspaces=true,
  numbers=left,
  numbersep=5pt,
  showspaces=false,
  showstringspaces=false,
  showtabs=false,
  tabsize=2
}

%"mystyle" code listing set
\lstset{style=mystyle}

\title{No Title}
\author{Claudeon R Susanto}
\date{}
\usepackage[T1]{fontenc}
\usepackage[utf8]{inputenc}
\usepackage[english]{babel}
\usepackage{lmodern}

\renewcommand{\familydefault}{\sfdefault}   % Supprime le serif (dyslexie)
\usepackage[font=sf, labelfont={sf}]{caption}
\usepackage{multicol}
\usepackage{makecell}
\renewcommand\theadalign{bc}
\renewcommand\theadfont{\bfseries}
\renewcommand\theadgape{\Gape[4pt]}
\renewcommand\cellgape{\Gape[4pt]}



% own commands
\newcommand{\eg}[0]{\textit{e.g. }}
\newcommand{\ie}[0]{\textit{i.e. }}
\newcommand{\impt}[0]{\textcolor{red}{\textbf{[IMPT] }}}


\renewcommand\thesubsection{\thesection.\arabic{subsection}}
\setlength{\columnseprule}{1pt}
\begin{document}
%\maketitle
\fontfamily{lmss}\selectfont
\begin{multicols}{2}
\section{R Programming}
\subsection*{List}
\begin{itemize}
	\item \texttt{[[idx]]}: get element in a list
	\item \texttt{str(ls)}: get \textbf{str}ucture of a list (similar to summary)
	\item \texttt{saveRDS} and \texttt{loadRDS}
	\item \texttt{unlist}: convert list to vector \impt
\end{itemize}
\subsection*{Recycling Rule}
\begin{itemize}
	\item shorter vectors are recycled until they match the length of the longest vector
	\item the length of the longest vector must be a multiple of the shorter vector in arithmetic operations!
\end{itemize}
\subsection*{Useful functions}
\begin{itemize}
	\item \texttt{sample(x, size, replace, prob)}
	\begin{itemize}
		\item \texttt{size}: length of output vector
		\item \texttt{replace}: if \texttt{TRUE}, then sampling is with replacement
		\item \texttt{prob}: a vector of probability weights
	\end{itemize}
	\item \texttt{any(duplicated(vec))}: returns true or false if there are any duplicated elements in a vector
	\item \texttt{rep(x, times, length.out)}
	\item \texttt{table()}
	\item \texttt{args(func)}: list the arguments of a function
	\item \texttt{seq(from, to, by, length)}
	\item \texttt{paste(v1, v2, sep)}: concatenate vectors after converting them to characters
	\begin{itemize}
		\item \texttt{sep}: separator between elements of \texttt{v1} and \texttt{v2}
		\item The recycling rule applies when \texttt{length(v1) != length(v2)}
	\end{itemize}
	\item \texttt{apply} function family: apply function to each row (1) or column (2)
	\begin{itemize}
		\item \texttt{apply(X, margin, func, ...)}
		\begin{itemize}
			\item Note that \texttt{X} must be a \textbf{matrix} or \textbf{df} in \texttt{apply}
		\end{itemize}
		\item \texttt{sapply} returns a vector or a matrix, \textbf{input must be 1 dimensional!}
		\item \texttt{lapply} returns a list, useful when the output of the function may not be all of the same length/type, \textbf{input must be 1 dimensional!}
		\item \texttt{replicate(n, func)}: replicate anonymous function $n$ number of times (especially useful for random number generations)
		\item \texttt{tapply()}: used to apply function and then group them into a table using grouping index
		\item \texttt{mapply(func, arg1, arg2, arg3, ...)}: like \texttt{sapply} but takes multiple vectors containing arguments to \texttt{func}
	\end{itemize}
\end{itemize}
\subsection*{Function debugging}
\begin{itemize}
	\item \texttt{cat("...")}: used to print statements
	\item \texttt{browser()}: debugging with breakpoint
\end{itemize}
\subsection*{Important classes}
\subsubsection*{Strings}
\begin{itemize}
	\item Start by importing \texttt{tidyverse} and \texttt{stringr}
	\item Library functions
	\begin{itemize}
		\item \texttt{str\_length}: returns vector of string lengths
		\item \texttt{str\_c(..., sep)}: concatenate strings with optional separator
		\item \texttt{str\_sub(string, start, end)}: returns vector of substrings
	\end{itemize}
	\item Regular expressions (\texttt{str\_view()} to test out regex), \href{https://stringr.tidyverse.org/articles/regular-expressions.html}{\textit{\underline{Tidyverse Article}}}
	\begin{itemize}
		\item to match an \textbf{a} at the beginning of a string
		\begin{verbatim}
			str_view(x, "^a")
		\end{verbatim}
		\item to match an \textbf{a} at the end of a string
		\begin{verbatim}
			str_view(x, "a$")
		\end{verbatim}
		\item to match an \textbf{a} or \textbf{e} at the end of a string
		\begin{verbatim}
			str_view(x, "[ae]$")
		\end{verbatim}
		\item to match a string of 3 chars with \textbf{a} in the middle
		\begin{verbatim}
			str_view(x, ".a.")
		\end{verbatim}

	\end{itemize}
	\item \texttt{str\_detect(vec, regex)}: returns a boolean vector
	\begin{itemize}
		\item | : means or
		\begin{verbatim}
			str_detect(street_names, "Jurong|Boon Lay")
		\end{verbatim}
		\item + : means modifier (pattern detected 1 or more times)
		\item (): to group stuff
		\item \texttt{$\backslash$$\backslash$w}: any word
		\item \texttt{[0-9]}: can be 0 to 9
		\item \texttt{$\backslash$$\backslash$d}: any number
		\begin{itemize}
			\item \texttt{$\backslash$$\backslash$d\{3,6\}} to search for digits repeating between 3 and 6 times
		\end{itemize}
		\item \impt \texttt{?about\_search\_regex} for help
		\item \impt \texttt{?base::regex} :help for regex from R base package; \texttt{[:punct:]}, \texttt{[:digit:]}, \texttt{[:space:]}
	\end{itemize}
	\item \texttt{str\_extract(vec, regex)}: returns a vector of strings, particularly helpful for \texttt{".a."} regex
	\begin{lstlisting}[language=R]
# To find the number of eggs given a sentence
str_extract(sent, "[0-9]+(?= eggs)")
# ?= is a look behind operator
# ?<= is a look ahead operator
	\end{lstlisting}
	\item \texttt{str\_trim}: to trim trailing whitespaces
	\item \texttt{str\_split}
	\item \texttt{str\_replace}
	\begin{lstlisting}[language=R]
# to remove duplicate words
str_replace(sent_type, "\\b(\\w+)\\b \\1", "\\1")\end{lstlisting}
Note that \texttt{$\backslash$$\backslash$b} means word boundary and \texttt{$\backslash$$\backslash$1} means group boundary 1
\item \texttt{str\_match}
\end{itemize}
\impt USE \texttt{vignette('stringr')} and \texttt{vignette('regular-expressions')} for help
\begin{itemize}
	\item \texttt{devtools::install\_github("gadenbuie/regexplain")} to install regexplain GUI, need to install \texttt{devtools} library first
	\item Also Tools $\rightarrow$ Addins $\rightarrow$ Browse Addins.. $\rightarrow$ regexplain (cheatsheet/GUI)
\end{itemize}
\subsubsection*{Factors}
\texttt{factor(vec, levels=c(...))}: convert \texttt{vec} to factors with fixes levels\\
\texttt{unique(vec)}: returns a vector with unique values

\subsubsection*{Date}
\begin{itemize}
	\item \impt \texttt{?strftime} for help page
	\item \texttt{as.Date(x, format)}: convert string x to \texttt{Date} object \\ e.g. \texttt{as.Date("2014/02/22", "\%Y/\%m/\%d")}
	\item \texttt{months(d)}: what month of the year is the date in?
	\item \texttt{weekdays(d)}: what day of the week is the date on?
	\item \texttt{Sys.Date()}
	\item \texttt{cut(x, breaks, labels)}: usually used to group dates that fall into a month/week/quarter
	\begin{itemize}
		\item \texttt{breaks}: numeric vector/string (\texttt{"month", "week"})
		\item \texttt{labels}: if \texttt{TRUE}, return a label vector
	\end{itemize}
\item \texttt{seq(d,d+365,by="1 week" or "1 quarter")}

\end{itemize}

\subsection*{Basic Plotting}
\subsubsection*{\texttt{plot()}}
\begin{itemize}
	\item \texttt{pch}: abbr. for plotting character
	\begin{lstlisting}[language=R]
	# show all pch characters
	example(pch)\end{lstlisting}
	\item \texttt{col}:
	\begin{lstlisting}[language=R]
	# show all preset colours
	colours()
	# set custom colour, alpha is transparency
	col <- rgb(..., alpha=?)\end{lstlisting}
	\item \texttt{cex}: abbr. for character expansion
	\item \texttt{bty}: change box borders
	\item \impt \texttt{?par} shows all parameters for \texttt{plot()}
	\item use \texttt{points()} or \texttt{lines()} to add more stuff to an existing plot
	\begin{itemize}
		\item \texttt{segments(x\_)}
	\end{itemize}

\end{itemize}
\subsubsection*{\texttt{barplot()}}
\begin{itemize}
	\item \texttt{horiz=TRUE} flip y and x axes
	\item \texttt{las} (under \texttt{?par})
	\item \texttt{par()}: \impt lists all the default parameters for plots (\texttt{mar}, \texttt{mfrow} etc.)
	\item How to set graphical param?
	\begin{lstlisting}[language=R]
# 1 row 2 columns plot
opar <- par(mfrow=c(1,2))
# plot some stuff
par(opar) # to set it back to default
	\end{lstlisting}
\end{itemize}
\subsubsection*{\texttt{hist()}}
\begin{itemize}
	\item \texttt{freq}: makes the y-axis a proportion of all the total shit (count/total), not total count using integer
\end{itemize}
\section{Stringr}
\textbf{(to convert numeric to string) Fixed vs scientific format}
\begin{itemize}
	\item Scientific: \texttt{1.989e+30} to denote $10^30$
	\item \texttt{format(x, scientific=TRUE)} to format number to string by specifying digit numbers etc.\\
	\impt \texttt{digits=} will format the smallest number so that it only has the specified significant digit, and other numbers in the vector follows
	\begin{lstlisting}[language=R]
	format(c(0.0011, 0.011, 1), digits=1)
	> [1] "0.001" "0.011" "1.000"
	\end{lstlisting}
	\item \texttt{formatC(x, format="f" OR "e" or "g")}\\
	\texttt{f} stands for fixed, \texttt{e} for scientific, and \texttt{g} for scientific if it saves space
\end{itemize}
\textbf{Stringr functions}
\begin{itemize}
	\item \texttt{str\_c}: concatenate like \texttt{paste}
	\item \texttt{str\_length}: find length
	\item \texttt{str\_sub}
	\item \texttt{str\_detect}: returns boolean vectors
	\item \texttt{str\_subset}:
	\item \texttt{str\_count}
	\item \texttt{str\_split}: \texttt{n=} returns maximum number of n elements, \texttt{simplify=} returns a matrix
	\begin{itemize}
		\item \impt \texttt{type=boundary("sentence")}
	\end{itemize}
	\item \texttt{str\_match}: returns a matrix with the capture or () regex
	\item \texttt{str\_to\_upper()}: returns a vec with all uppercase elements
	\item \texttt{str\_to\_lower()}
	\item \texttt{regex(expr, ignore\_case = TRUE)}: tells regex to ignore case
\end{itemize}
\textbf{Rebus package}
\begin{itemize}
	\item \texttt{install.packages("rebus")} $\Rightarrow$ \texttt{library(rebus)}
	\item \texttt{rebus} syntax can be used for stringr pattern instead of regex
	\begin{lstlisting}[language =R]
		pattern = START %R% "a"
		# strings that start with "a"
		# same as regex "^a"
		# END is also possible
		# %R% is read as 'then'\end{lstlisting}
	\item \texttt{ANY\_CHAR}
	\item \texttt{WRD}: word, \texttt{SPC}: Space
	\begin{lstlisting}[language=R]
	# to capture word ending in ING
	one_or_more(WRD) %R% "ING"
	# equals to \w+ING
	\end{lstlisting}
	\item \texttt{or(p1, p2)}: kinda like \texttt{|} in regex
	\begin{itemize}
		\item \texttt{or1(vec)}: pass vec as alternatives instead of arguments
	\end{itemize}
	\item \texttt{char\_class("Aa")}: kinda like \texttt{"[Aa]"} in regex
	\item \texttt{negated\_char\_class("aiueoAIUEO")}: self-explanatory
	\item \texttt{optional()}: \texttt{?} in regex
	\item \texttt{zero\_or\_more()}: \texttt{*} in regex
	\item \texttt{one\_or\_more()}: \texttt{+} in regex
	\item \texttt{repeated()}: \texttt{\{m,n\}} in regex
	\item \texttt{exactly()}: matches exact string
	\item \texttt{capture(pattern)}: group parts of pattern together, which is () in regex format\\
	*use \texttt{REF1}, \texttt{REF2}, \texttt{REF3} to refer to the capture group (exact match) which is \texttt{$\backslash$1}, \texttt{$\backslash$2} and so on in regular regex

\end{itemize}
\textbf{Stringi functions}
\begin{itemize}
	\item \texttt{stri\_isempty()}: returns boolean
\end{itemize}
\textbf{Miscellaneous}
\begin{itemize}
	\item \texttt{strftime(date, format)}: string from time object
	\item \texttt{as.POSIXct(date\_string, format)}: convert string to Date time
\end{itemize}


\section{R Markdown (RMD)}
\begin{itemize}
	\item \texttt{.yaml} header
	\begin{lstlisting}[language=R]
	title:"..."
	output:
		html-document:
		toc: true #table of content
		toc_float: true # floating TOC at the left side of the window
			collapsed: true
			smooth_scroll: true
		toc_depth: 2
		number_sections:true/false
	date: `r format(Sys.time(), "%d %B %Y")`
	params:
		country: Indonesia
	\end{lstlisting}
\begin{itemize}
	\item how to reference?? $\Rightarrow$ I want die liao \texttt{`r params\$country`}
	\item Referencing is important as it allows more control over the report, don't need to manually change the name of every variable if we want something else
\end{itemize}

	\item R Setup \impt , will apply settings globally
	\begin{lstlisting}[language=R]
	```{r setup, include=FALSE}
	knitr::opts_chunk$set(fig.align='center', echo=TRUE)
	```\end{lstlisting}
	\item Use \texttt{`r var`} to insert inline code and ask R to run it
	\item Figure
	\begin{itemize}
		\item \texttt{include=FALSE/TRUE}: to include the output or not
		\item \texttt{fig.width}, \texttt{fig.height}, \texttt{fig.dim = c(w,h)}, \texttt{out.width="XX\%"}
		\item \texttt{fig.align='left'/'centre'}
		\item \texttt{fig.cap} for captions
	\end{itemize}
	\item Bulleted list: just indent and use '-'
	\item Dsiplay table: use \texttt{kable(df, col.names=c(...))}
	\begin{itemize}
		\item Important parameters: \texttt{caption}, \texttt{align="ccc" or "lll"} for text alignment inside boxes
	\end{itemize}
\end{itemize}

\subsection*{Code Chunk Settings}
\begin{itemize}
	\item \texttt{include=FALSE} doesn't print the code
	\item \texttt{echo=FALSE} usually for plots, don't include the actual code but just runs it
	\item \texttt{eval=FALSE} code chunk is not run/evaluated
	\item \texttt{collapse=TRUE} combines text output and source code in single block
	\item \texttt{message=FALSE}
	\item \texttt{warning=FALSE}
	\item \texttt{error=TRUE} will continue to knit the file even when there are errors and will include error messages in the file
\end{itemize}

\section{Importing Data}

\impt use \texttt{read.delim} or \texttt{readLines} if none is working

\subsection*{CSV Files}
\texttt{read.csv()}: main arguments:
\begin{itemize}
	\item \texttt{file}: filename/path
	\item \texttt{skip}: skip lines?
	\item \texttt{header}: default is \texttt{TRUE}
	\item \texttt{row.names}
	\item \texttt{stringsAsFactors}
	\item \texttt{na.strings}: what are the \texttt{NA} values
	\item \texttt{colClasses}: what classes are the columns (in terms of class names vector)\\
\end{itemize}
\textbf{Procedure when dealing with CSV}:
\begin{itemize}
	\item \texttt{apply(salaries, 2, function(x) sum(is.na(x)))} \impt (check if any column has missing values)
	\item if \texttt{read.csv} doesn't work, can try \texttt{readLines} and \texttt{str\_split} to split commas
\end{itemize}


\subsection*{Excel Files}
\begin{itemize}
	\item import \texttt{readxl}, data is in the form of a tibble
	\item \texttt{read\_excel(path, sheet=?)}: \texttt{sheet} parameter can be string or integer
	\item \texttt{sheet\_names(path)}: to retrieve sheet names
\end{itemize}

\subsection*{JSON Files}
\begin{itemize}
	\item import \texttt{jsonlite}
	\item \texttt{fromJSON(txt)}: takes up text/string object as an argument
	\item \texttt{readLines(path)}: returns a string \impt line break will count as another element of a vector
	\item \texttt{prettify()}
	\item \impt How to convert list to data frame?
	\begin{enumerate}
		\item create a function \texttt{ls\_to\_df} which returns \texttt{data.frame} given an element of a list
		\item \texttt{lapply} the list to return a list of dataframes
		\item use \texttt{do.call} to combine the individual dataframes into one single dataframe
		\begin{lstlisting}[language=R]
df_row_list <- lapply(list, ls_to_df)
# combine repeatedly
do.call(rbind, df_row_list)\end{lstlisting}
	\end{enumerate}
	\item Some thoughts \impt Are there missing data for any observation?? if yes then remove
\end{itemize}
\subsection{OOP in R}
\impt \underline{Main purpose}: call function the same way (with similar syntax but different behaviour for each class) \eg \texttt{plot} works differently for timeseries and vectors
\textbf{S3 classes}
\begin{itemize}
	\item \texttt{methods}: to search for available methods
	\item \texttt{summary}
\end{itemize}
\begin{lstlisting}[language=R]
studentBio <- list(studentName = "Harry Potter", studentAge = 19, studentContact="London")
class(studentBio) <- "StudentInfo"

# how to assign method
contact <- function(object) {
	UseMethod("contact")
}
contact.StudentInfo <- function(object) {
	cat("Your contact is", object$studentContact, "\n")
}
# can just call contact(studentBio) without .StudentInfo
\end{lstlisting}
\textbf{S4 classes}
\begin{lstlisting}[language=R]
# How to set Class with slots
setClass("employee", slots=list(name="character", id="numeric", contact="character"))

# Constructor
obj <- new("employee",name="Steven", id=1002, contact="West Avenue")
\end{lstlisting}

\impt How to add method?
\begin{lstlisting}[language=R]
setMethod("show",
signature(object="employee"),
definition=function(object) {
	# do stuff
})
\end{lstlisting}

\impt Tips for dealing with S4 data
\begin{itemize}
	\item \texttt{isS4(obj)}: check if obj is S4
	\item \texttt{slotNames(obj)} list all the attributes/slots
	\item \texttt{methods(class="????")}: to list out all the methods\\
	\texttt{methods(generic.function="plot")}: to list out all the classes a method can be applied to
	\item \texttt{vignette("class")}: for documentation
\end{itemize}
\textbf{RC classes}
\section{Databases}
\textbf{How to connect?}
\begin{itemize}
	\item Install the requisite package on R
	\item Authenticate to the database server
	\item Query/Extract the data
	\item Analyse the data
	\item Close the connection
\end{itemize}
\subsection{MongoDB}
\textbf{Steps to connect}
\begin{itemize}
	\item \impt \href{https://www.mongodb.com/docs/v5.0/tutorial/getting-started}{MongoDB Tutorial Docs}
	\item Code to connect
	\begin{lstlisting}[language=R]
library(mongolite)
library(jsonlite)
# eXXXXX:pwd
credentials <- paste0(readLines("mongo_user_pwd.txt", warn=FALSE), collapse=":")
connection_string <- paste0("mongodb://",credentials, "@rshiny.nus.edu.sg:2717/test")
con2 <- mongo(verbose=TRUE, collection="restaurants", url=connection_string)
con2$count()
	\end{lstlisting}
\end{itemize}
\subsection{Data from Web}

\subsubsection{Download File from Link}
\begin{itemize}
	\item how to download
	\begin{lstlisting}[language=R]
		imda_url <- "https://data.gov.sg/dataset/02c1f624-489f-40ad-8fdd-5e66e46b2722/download"
		return_val <- download.file(imda_url, "../data/imda_data.zip")
		con <- unz("../data/imda_data.zip", "wage-02-size2-annual.csv")
		wages_data <- read.csv(con, header=TRUE)
	\end{lstlisting}
	\item \texttt{download.file()}, \texttt{mode="wb"} for Windows
	\item \texttt{unz}: to unzip
\end{itemize}
\subsubsection{Developer API}
\begin{itemize}
	\item Normal browser $\xleftrightarrow[response]{request}$ Web server
	\item request data from server that is continuously running
	\item \impt Usually for Real-time data
	\item how to get data?
	\begin{lstlisting}[language=R]
library(httr)
set_config(verbose())
url <- "https://api.data.gov.sg/v1/transport/taxi-availability"
taxi_avail <- GET(url, query=list(date_time="2022-08-01T09:00:00"))

taxi_data <- content(taxi_avail)
	\end{lstlisting}
\end{itemize}

\end{multicols}






\end{document}