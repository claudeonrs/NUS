\documentclass{article}
\usepackage[a4paper, left=15mm, top=20mm, right=15mm,bottom=20mm]{geometry}
\usepackage{amsmath, amssymb, amsfonts}
\usepackage{fancyhdr}
\usepackage{graphicx}
\graphicspath{ {./images/} }
\usepackage{float}
\usepackage{hyperref}
\usepackage{lscape}
%\usepackage{arev}

\pagestyle{fancy}
\fancyhf{}
\lhead{SP1541}
\rhead{claudeonrs}
\rfoot{\thepage}
\usepackage{amsmath, amssymb, amsfonts, listings}
\usepackage{xcolor}
\usepackage{enumitem}
\setlist{nolistsep}


%New colors defined below
\definecolor{codegreen}{rgb}{0,0.6,0.4}
\definecolor{codegray}{rgb}{0.5,0.5,0.5}
\definecolor{codepurple}{rgb}{0.58,0,0.82}
\definecolor{backcolour}{rgb}{0.95,0.95,0.92}
\definecolor{commentgreen}{rgb}{0.4,0.8,0.6}
%Code listing style named "mystyle"
\lstdefinestyle{mystyle}{
  backgroundcolor=\color{backcolour},   
  commentstyle=\color{red},
  keywordstyle=\color{blue},
  numberstyle=\tiny\color{codegray},
  stringstyle=\color{codegreen},
  basicstyle=\ttfamily,
  breakatwhitespace=false,         
  breaklines=true,                 
  captionpos=b,                    
  keepspaces=true,                 
  numbers=left,                    
  numbersep=5pt,                  
  showspaces=false,                
  showstringspaces=false,
  showtabs=false,                  
  tabsize=2
}

%"mystyle" code listing set
\lstset{style=mystyle}

\title{No Title}
\author{Claudeon R Susanto}
\date{}
\usepackage[T1]{fontenc}
\usepackage[utf8]{inputenc}
\usepackage[english]{babel}
\usepackage{lmodern}

\renewcommand{\familydefault}{\sfdefault}   % Supprime le serif (dyslexie)
\usepackage[font=sf, labelfont={sf}]{caption}
\usepackage{multicol}
\usepackage{makecell}
\renewcommand\theadalign{bc}
\renewcommand\theadfont{\bfseries}
\renewcommand\theadgape{\Gape[4pt]}
\renewcommand\cellgape{\Gape[4pt]}

\renewcommand\thesubsection{\thesection.\arabic{subsection}}
\setlength{\columnseprule}{1pt}
\begin{document}
%\maketitle
\fontfamily{lmss}\selectfont
\begin{multicols}{2}
\section{NUS Libraries Online Tutorials}

Types of documents
\begin{itemize}
	\item Thesis \& Dissertations, conference proceedings, journal \& news articles, patents
	\item Review articles (good for summarizing recent developments/if u're new to the topic), bibliographies, books
\end{itemize}

Search in multiple platform to avoid info from falling through the crack

\textbf{NUS Guides}:
\begin{itemize}
	\item \textbf{Subject guides}: guide avail to NUS community for specific subject areas \href{https://libguides.nus.edu.sg/?sg=s}{\textbf{[Link]}}
	\item \textbf{Other guides}: APA Citation Style, Zotero, patents, how to find free online content! \href{https://libguides.nus.edu.sg/?b=t}{\textbf{[Link]}}
\end{itemize}


\section{Week 2}
\subsection{Tutorial 2.1}
\textbf{Current problems with scientific communication}
\begin{itemize}
	\item Current media and its audience value \underline{speed} and \underline{ease of digestion} of information over quality and reliability
	\begin{itemize}
		\item Lack of transparency, people don't know what's happening as media leave out limitations and caveats, as well as scientific methodology due to journalistic constraints
	\end{itemize}
	\item Exaggerating/inflating information to generate more clicks, can be misused or exploited by media/authority
	\item Lack of respect from the general public towards the scientific community
	\begin{itemize}
		\item The uncertain nature of science $\rightarrow$ contradictory headlines/claims, people don't know what's happening
	\end{itemize}	 
	\item Difference in views (Lack of scientific literacy) between the layman and the scientist (e.g. links between vaccination and autism, does man contribute to global warming)
	\begin{itemize}
		\item Difference in view regarding contribution of science towards society $\rightarrow$ affects public policy and scientific progress
	\end{itemize}
	\item The public are generally intimidated by scientific jargons and abstract concepts
	\item Lack of scientific publications that aim to popularize science to the masses (at least in SG)\\
\end{itemize}

\textbf{Aims of scientific communication}
\begin{itemize}
	\item Educate public on current scientific developments and its relevance to society
	\begin{itemize}
		\item Obligation to be transparent regarding science work as science uses large amounts of resources
	\end{itemize}
	\item Spark meaningful debates and discussion
	\item Increase interest in science and allow people to make more informed decisions as well as political decisions
	\item Fusion of public and scientific values (general public have more scientific values such as accuracy and reproducibility etc.)\\
\end{itemize}

\textbf{Why is scientific communication useful for scientists?}
\begin{itemize}
	\item Allow scientists to discuss different ideas
	\begin{itemize}
		\item especially scientists from different domains as even an expert in one area might be an amateur in other areas
	\end{itemize}
	\item Realize the relevance and societal impact in their work
	\begin{itemize}
		\item Clarify the aim of their work through writing
	\end{itemize}
	\item A reflection of their knowledge and how much they have learnt from their studies
	\item Wider social perspective
	\begin{itemize}
		\item Thinking from general public perspective
		\item Deal with different perspectives and learn how to explain abstract concepts to the layman\\
	\end{itemize}
\end{itemize}

\textbf{Color and Clarity}: purpose of scientific communication!
\textbf{Some strategies (Talia Gershon)}
\begin{itemize}
	\item different audience? get a sense of audience's prior knowledge by asking questions
	\item everyday object (noise cancelling headphones)
	\item how does this affect them (significance) on personal level
\end{itemize}

\subsection{Tutorial 2.2}

\section{Week 3}
\subsection{Tutorial 3.1}
\subsection{Tutorial 3.2}
\section{Week 4}
\subsection{Tutorial 4.1}
\subsection{Tutorial 4.2}
\section{Week 5}
\subsection{Tutorial 5.1}
\subsection{Tutorial 5.2}
\section{Week 6}
\subsection{Tutorial 6.1}
\subsection{Tutorial 6.2}
\end{multicols}






\end{document}