\documentclass{article}
\usepackage[a4paper, left=15mm, top=20mm, right=15mm,bottom=20mm]{geometry}
\usepackage{amsmath, amssymb, amsfonts}
\usepackage{fancyhdr}
\usepackage{graphicx}
\graphicspath{ {./images/} }
\usepackage{float}
\usepackage{hyperref}
\usepackage{lscape}
%\usepackage{arev}

\pagestyle{fancy}
\fancyhf{}
\lhead{SP1541}
\rhead{claudeonrs}
\rfoot{\thepage}
\usepackage{amsmath, amssymb, amsfonts, listings}
\usepackage{xcolor}
\usepackage{enumitem}
\setlist{nolistsep}


%New colors defined below
\definecolor{codegreen}{rgb}{0,0.6,0.4}
\definecolor{codegray}{rgb}{0.5,0.5,0.5}
\definecolor{codepurple}{rgb}{0.58,0,0.82}
\definecolor{backcolour}{rgb}{0.95,0.95,0.92}
\definecolor{commentgreen}{rgb}{0.4,0.8,0.6}
%Code listing style named "mystyle"
\lstdefinestyle{mystyle}{
  backgroundcolor=\color{backcolour},
  commentstyle=\color{red},
  keywordstyle=\color{blue},
  numberstyle=\tiny\color{codegray},
  stringstyle=\color{codegreen},
  basicstyle=\ttfamily,
  breakatwhitespace=false,
  breaklines=true,
  captionpos=b,
  keepspaces=true,
  numbers=left,
  numbersep=5pt,
  showspaces=false,
  showstringspaces=false,
  showtabs=false,
  tabsize=2
}

%"mystyle" code listing set
\lstset{style=mystyle}

\title{No Title}
\author{Claudeon R Susanto}
\date{}
\usepackage[T1]{fontenc}
\usepackage[utf8]{inputenc}
\usepackage[english]{babel}
\usepackage{lmodern}

\renewcommand{\familydefault}{\sfdefault}   % Supprime le serif (dyslexie)
\usepackage[font=sf, labelfont={sf}]{caption}
\usepackage{multicol}
\usepackage{makecell}
\renewcommand\theadalign{bc}
\renewcommand\theadfont{\bfseries}
\renewcommand\theadgape{\Gape[4pt]}
\renewcommand\cellgape{\Gape[4pt]}

\renewcommand\thesubsection{\thesection.\arabic{subsection}}
\setlength{\columnseprule}{1pt}
\begin{document}
%\maketitle
\fontfamily{lmss}\selectfont
\begin{multicols}{2}
\section{NUS Libraries Online Tutorials}

Types of documents
\begin{itemize}
	\item Thesis \& Dissertations, conference proceedings, journal \& news articles, patents
	\item Review articles (good for summarizing recent developments/if u're new to the topic), bibliographies, books
\end{itemize}

Search in multiple platform to avoid info from falling through the crack

\textbf{NUS Guides}:
\begin{itemize}
	\item \textbf{Subject guides}: guide avail to NUS community for specific subject areas \href{https://libguides.nus.edu.sg/?sg=s}{\textbf{[Link]}}
	\item \textbf{Other guides}: APA Citation Style, Zotero, patents, how to find free online content! \href{https://libguides.nus.edu.sg/?b=t}{\textbf{[Link]}}
\end{itemize}
\subsection{Search strategies}
\textbf{Challenges in searching}:
\begin{itemize}
	\item may miss important literature
	\item how to find the right keywords?
	\item not specific enough
	\item irrelevant papers
	\item publications not recent
\end{itemize}
\textbf{How to search?}:
\begin{enumerate}
	\item Identify the keywords
	\item Add search operators
	\begin{itemize}
		\item \texttt{OR}: eg COVID19 OR coronavirus
		\item \texttt{*}: wildcard eg COVID* $\rightarrow$ COVID19/COVID-19/COVID\\
		virus* $\rightarrow$ viruses/virus (singular/plural)
		\item \texttt{AND}
		\item \texttt{" "} search for EXACT phrases
	\end{itemize}
	\item Refine the search statement
	\begin{itemize}
		\item look at the search results and articles
		\item are there new useful keywords/synonyms?
		\item are there any irrelevant articles? $\rightarrow$ remove those noise keywords
	\end{itemize}
\end{enumerate}
\subsection{Where to search?}
\textbf{FindMore}
\begin{itemize}
	\item Books and E-resources $\Rightarrow$ View more
	\item Sort by relevance/date/author
	\item \textbf{Refine your search}
	\begin{itemize}
		\item Content type (journal article etc.)
		\item Publication date (1,3,5 years etc.)
	\end{itemize}
    \item View abstract/summary to not waste time reading article
    \item Colourful doughnut: usage of article
    \item \textbf{Cites/Cited by}: show a list of references which have been referred to by the particular publication
    \item \textbf{Advanced search}: search by discipline etc.
    \item \textbf{Save search}: allows users to save search
\end{itemize}
\textbf{Web of Science}
\begin{itemize}
	\item Peer reviewed journals!
	\item NUS Libraries Portal $\rightarrow$ Databases $\rightarrow$ Web of Science
	\item Filter by:
	\begin{itemize}
		\item Publication years
		\item Document type (\textbf{[IMPT]} NO REVIEW ARTICLES)
		\item Sort: by \textbf{number of citations, relevance}
	\end{itemize}
    \item Read abstract
    \item \textbf{Citations}
    \begin{itemize}
    	\item Cited by
    	\item Cited papers
    \end{itemize}
    \item Find it! @ NUS Library
    \item Analyze results
    \begin{itemize}
    	\item Discover insights on areas as well as authors
    \end{itemize}
    \item History
    \
\end{itemize}
\textbf{Factiva}
\begin{itemize}
	\item International news database produced by Dow Jones
	\item Access from NUS Libraries Portal
\end{itemize}
\subsection{A.R.T Evaluation Criteria}
\begin{itemize}
	\item Authoritative
	\begin{itemize}
		\item Who is the author(s)?
		\item What are the author(s) credentials or organization affiliations?
		\item Has the author(s) published widely? Is the author an established expert in the field?
		\item Is the information from original and reliable authentic sources? (Does the URL reveal anything about the source?)
		\item Is there contact information, such as email address?
	\end{itemize}
	\item Relevance
	\begin{itemize}
		\item Is the information relevant to your research topic?
		\item Who are the intended audience of the information?
		\item Is the information at the appropriate level (i.e. not too elementary or overly advanced for your needs)?
	\end{itemize}
	\item Timely
	\begin{itemize}
		\item Is the information updated?
		\item When was the information published?
		\item Is the information up to date for the topic?
		\item When was the information last updated or revised?
	\end{itemize}
\end{itemize}

\subsection{Alternative ways to access full texts}
\textbf{Google Scholar}
\begin{itemize}
	\item use similar keyword search
	\item Library links $\rightarrow$ Search NUS $\rightarrow$ Open WorldCat - Library Search \& National University of Singapore - FindIt!
\end{itemize}
\textbf{Proxy Bookmarklet}
\begin{itemize}
	\item \href{https://libguides.nus.edu.sg/findfulltext/proxybookmark}{How to install NUS Proxy Bookmarklet}
\end{itemize}


\section{Week 2}
\subsection{Tutorial 2.1}
\textbf{Current problems with scientific communication}
\begin{itemize}
	\item Current media and its audience value \underline{speed} and \underline{ease of digestion} of information over quality and reliability
	\begin{itemize}
		\item Lack of transparency, people don't know what's happening as media leave out limitations and caveats, as well as scientific methodology due to journalistic constraints
	\end{itemize}
	\item Exaggerating/inflating information to generate more clicks, can be misused or exploited by media/authority
	\item Lack of respect from the general public towards the scientific community
	\begin{itemize}
		\item The uncertain nature of science $\rightarrow$ contradictory headlines/claims, people don't know what's happening
	\end{itemize}
	\item Difference in views (Lack of scientific literacy) between the layman and the scientist (e.g. links between vaccination and autism, does man contribute to global warming)
	\begin{itemize}
		\item Difference in view regarding contribution of science towards society $\rightarrow$ affects public policy and scientific progress
	\end{itemize}
	\item The public are generally intimidated by scientific jargons and abstract concepts
	\item Lack of scientific publications that aim to popularize science to the masses (at least in SG)\\
\end{itemize}

\textbf{Aims of scientific communication}
\begin{itemize}
	\item Educate public on current scientific developments and its relevance to society
	\begin{itemize}
		\item Obligation to be transparent regarding science work as science uses large amounts of resources
	\end{itemize}
	\item Spark meaningful debates and discussion
	\item Increase interest in science and allow people to make more informed decisions as well as political decisions
	\item Fusion of public and scientific values (general public have more scientific values such as accuracy and reproducibility etc.)\\
\end{itemize}

\textbf{Why is scientific communication useful for scientists?}
\begin{itemize}
	\item Allow scientists to discuss different ideas
	\begin{itemize}
		\item especially scientists from different domains as even an expert in one area might be an amateur in other areas
	\end{itemize}
	\item Realize the relevance and societal impact in their work
	\begin{itemize}
		\item Clarify the aim of their work through writing
	\end{itemize}
	\item A reflection of their knowledge and how much they have learnt from their studies
	\item Wider social perspective
	\begin{itemize}
		\item Thinking from general public perspective
		\item Deal with different perspectives and learn how to explain abstract concepts to the layman\\
	\end{itemize}
\end{itemize}

\textbf{Color and Clarity}: purpose of scientific communication!
\textbf{Some strategies (Talia Gershon)}
\begin{itemize}
	\item different audience? get a sense of audience's prior knowledge by asking questions
	\item everyday object (noise cancelling headphones)
	\item how does this affect them (significance) on personal level
	\item Storytelling (make this relatable to them [hook])
\end{itemize}

\subsection{Tutorial 2.2}

\textbf{How are papers organized?}
\begin{itemize}
	\item Title (important \textbf{keywords}) $\rightarrow$ abstract (summary)
	\item Introduction (what was the \underline{problem}? what was the \underline{reason} for the research? what have previous studies done? what are the \underline{hypotheses}?)
	\item Methods (how readers can replicate the research [\underline{procedures/methodology, observations/data}])
	\item Results (how does it \textit{contribute} to the body of sci knowledge?)
	\item Discussion/conclusions
	\item Acknowledgements $\rightarrow$ references \\
\end{itemize}
\textbf{Start by asking \underline{IMRAD}}
\begin{itemize}
	\item \textbf{I}ntro: what was the question? why is it \textbf{I}mportant
	\item \textbf{M}ethods: how did the research try to answer it / solve the problem?
	\item \textbf{R}esults: what did they find?
	\item \textbf{A}nd \textbf{D}iscussion: what do the results mean? How does this contribute to the body of scientific knowledge
\end{itemize}
\textbf{How to read a scientific paper?}
\begin{itemize}
	\item \textbf{Skim} the article without taking notes (big picture)
	\item \textbf{Re-read} especially \textit{results} and \textit{methods}\\
	Try to interpret the data before reading explanations
	\item \textbf{Ask} questions
	\begin{itemize}
		\item What problems does the study address?
		\item Why is it important?
		\item Is the method good?
		\item Are the findings supported by evidence/other work in the field?
		\item Is the study \underline{repeatable}? How big is the sample size? Is this representative of the larger population?
		\item What variables were held constant?
	\end{itemize}
	\item \textbf{Write} a summary\\
\end{itemize}

\textbf{Other useful tips}
\begin{itemize}
	\item Draw inferences (rely on background knowledge)
	\item look for words (unexpected, in contrast to previous work, hypothesize, suggest) and main points
	\item take notes in own words (summary) and develop a template
\end{itemize}
\textbf{Other difficulties in reading papers}
\begin{itemize}
	\item Depends on the writing skills of the scientists involved
	\item Sometimes describes only the 'what' (methods, results), not the 'why'
	\item Paper has no clear structure
	\item Description of experiment is ambiguous
	\item Authors refer back to previous papers
	\item Authors firmly believe in their particular model, not open to criticisms
	\item Authors overstate the importance of their findings
\end{itemize}

\textbf{Criteria for news article}
\begin{itemize}
	\item Trivial assumptions?
	\item Generalizations? or can it only be applied to specific areas under certain circumstances?
	\item Specific fields only? or for general public?

\end{itemize}


\section{Week 3}
\subsection{Tutorial 3.1}
\textbf{Strategies (important for reflection)}
\begin{itemize}
	\item \textbf{Unexpectedness} Belief turned upside down (mindblown), example would be genes utilising humans in Dawkins chapter
	\item \textbf{Possibility}: possible to be used for more application
	\item \textbf{Potential/Effect/Impact}: Significance
	\item Functional recontextualization - use function without much details (simplification)
	\begin{itemize}
		\item Not talking about what it is but about what it does
	\end{itemize}
	\item Storytelling (plot, character that you can relate to)
	\item Descriptive (5 senses that you can be fully immersed in)
	\item Conversational tone (actually, of course, )
	\item Personal pronouns (I, we, you)
	\begin{itemize}
		\item 'I' used to indicate level of expertise to separate you and I
	\end{itemize}
	\item Analogy/metaphors
	\item Deontological appeal
	\begin{itemize}
		\item add an air of mystery/intrigue
	\end{itemize}
\end{itemize}
\subsection{Tutorial 3.2}
\textbf{Headlines/leads}:
\begin{itemize}
	\item \underline{move 1}: intro key findings
	\item \underline{move 2}: highlight/describe significance/impact\\
	concrete examples/impact $\Rightarrow$ research not just for the sake of science, but can be useful for the layman
	\item Combat popular perceptions regarding controversial stuff
\end{itemize}
\textbf{some more strategies}:
\begin{itemize}
	\item \underline{Definitions} (using brackets to explain the word/term)
	\item Analogy (metaphor is more like something ... is ...)\\
	Don't use unfamiliar concept to explain another unfamiliar concept (e.g. winter skid tyres vs olfactory bulbs)
	\item \underline{Descriptions} (explaining process/mechanism/concept)
\end{itemize}
\textbf{Areas of evaluation for DEONTOLOGICAL APPEAL}
\begin{itemize}
	\item Unexpectedness (mindblowing stuff)
	\item Possibility (do the results have potential impacts in multiple domains in the future?)
\end{itemize}
\section{Week 4}
\subsection{Tutorial 4.1}
\textbf{Prime numbers Sautoy}
\begin{itemize}
	\item Humanize mathematicians through narrative story
	\item Effect on readers: sense of betrayal through plot twists (April fool prank)
	\item Writing style: Narrative
	\item \textbf{Teleological appeal}:
	\begin{itemize}
		\item \textbf{APPLICATION/BENEFITS} FOR THE READERS
		\item but might not be the best thing?
	\end{itemize}
\end{itemize}
\subsection{Tutorial 4.2}
\textbf{Cicada shit}
\begin{itemize}
	\item Introduce the key finding
	\item Significance: "First mathematical treatment", ".. to solve the problem of .."
	\item \textbf{RATIONALE FOR DOING THE RESEARCH}: nice move
	\begin{itemize}
		\item \textbf{overcoming current limitations}
		\item \textbf{Addressing research gap} (gap in the literature/missing info/explanation)
	\end{itemize}
    \item \textbf{DIRECT QUOTE}: to $\uparrow$ credibility
    \begin{itemize}
    	\item Author
    	\item Credentials (job/specialty/institute/any awards/accolades)
    	\item Conversational/more personable
    	\item can be obtained from other news articles/past interview videos (use told)
    \end{itemize}
	\item \textbf{BACKGROUND INTRO}: need to refer to other research/news articles for background knowledge and cite them!
	\item \textbf{NON-TECHNICAL TERMS}
	\item \textbf{EXEMPLIFICATION}: giving specific examples
	\begin{itemize}
		\item Examples must also be understood by majority of readers
		\item e.g. everyday objects
	\end{itemize}
	\item \textbf{COHESIVE DEVICE}
	\begin{itemize}
		\item \textbf{Synonyms}: using different names for the same thing to reduce repetition
		\item Transition words: Although, However
		\item Demonstrative pronouns: this/that $\Rightarrow$ refer to complex concepts that have been explained beforehand
	\end{itemize}
    \item Coherence vs Cohesion
    \begin{itemize}
    	\item coherence: ensure logical flow and understandability
    	\item cohesion: all the little things come together to tell a whole story
    \end{itemize}
    \item Hedging:
    \begin{itemize}
    	\item Acknowledging limitations (there are assumptions)
    	\item Opening avenues for future research
    	\item alternative results
    \end{itemize}
\end{itemize}
\section{Week 5}
\subsection{Tutorial 5.1}
\textbf{Hungry Study}
\begin{itemize}
	\item Hunger $\Rightarrow$ anyone can relate $\Rightarrow$ relatable
	\item Move 6: introduced methods \& findings
	\begin{itemize}
		\item only included part of methodology that was relevant to understanding key finding
		\item easy to understand
	\end{itemize}
	\item Move: introduce key findings
	\item Tries to relate to readers because being 'hangry' is something that one has experienced before
	\begin{itemize}
		\item What happens to u also happen to alot of ppl
	\end{itemize}
    \item \textbf{Evaluate the findings}
    \begin{itemize}
    	\item acknowledge limitations of research and also previous research WHILE emphasising the validity/reliability of research
    	\begin{itemize}
    		\item incremental nature of science
    	\end{itemize}
    	\item Opening up future directions for research (call to action????)
    \end{itemize}
    \item Evaluative language: could, possibly etc
    \begin{itemize}
    	\item acknowledge limitations and assumptions
    	\item unethical research???? etc. unsound methodology?? but a peer-reviewed research paper should not have this kind of problem
    \end{itemize}
    \item \textbf{Conversational language}
    \begin{itemize}
    	\item More conversational transition words (in the meantime, at the same time)
    	\item inclusive pronouns
    \end{itemize}
    \item Katong Flower Shop, BYOP protein??, Central Narcotics Bureau, Yeo's,  %九鲜
\end{itemize}
\subsection{Tutorial 5.2}
\begin{itemize}
	\item Synonyms
	\item Coherence: logical flow of writing\\
	Why did the author use the data from Bengal Delta and apply it to Yemen??
	\textbf{Move 7: explanation of results}\\
	THey omitted some explanation and logical link (e.g. the researchers measured some variables but did not explain why they are correlated to cholera outbreak)
	\item Cohesion:
	\item borrowed objective POV from outside the research
\end{itemize}
\section{Week 6}
\subsection{Tutorial 6.1}
\subsection{Tutorial 6.2}
\section{Week 7}
\subsection{Tutorial 7.1}
\subsection{Tutorial 7.2}
\section{Week 8}
\subsection{Tutorial 8.1}
\subsection{Tutorial 8.2}
\section{Week 9}
\subsection{Tutorial 9.1}
\subsection{Tutorial 9.2}
\section{Week 10}
\subsection{Tutorial 10.1}
\subsection{Tutorial 10.2}
\section{Week 11}
\subsection{Tutorial 11.1}
\subsection{Tutorial 11.2}
\section{Week 12}
\subsection{Tutorial 12.1}
\subsection{Tutorial 12.2}

\end{multicols}






\end{document}