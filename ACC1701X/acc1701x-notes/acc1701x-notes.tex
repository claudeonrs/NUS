\documentclass{article}
\usepackage[a4paper, left=15mm, top=20mm, right=15mm,bottom=20mm]{geometry}
\usepackage{amsmath, amssymb, amsfonts}
\usepackage{fancyhdr}
\usepackage{graphicx}
\graphicspath{ {./images/} }
\usepackage{float}
\usepackage{hyperref}
\usepackage{lscape}
\usepackage{arev}

\pagestyle{fancy}
\fancyhf{}
\lhead{ACC1701X}
\rhead{claudeonrs}
\rfoot{\thepage}
\usepackage{amsmath, amssymb, amsfonts, listings}
\usepackage{xcolor}
\usepackage{enumitem}
\setlist[itemize]{noitemsep, topsep=0pt}
\setlist[enumerate]{noitemsep, topsep=0pt}
\setlist[description]{noitemsep, topsep=0pt}


%New colors defined below
\definecolor{codegreen}{rgb}{0,0.6,0.4}
\definecolor{codegray}{rgb}{0.5,0.5,0.5}
\definecolor{codepurple}{rgb}{0.58,0,0.82}
\definecolor{backcolour}{rgb}{0.95,0.95,0.92}
\definecolor{commentgreen}{rgb}{0.4,0.8,0.6}
%Code listing style named "mystyle"
\lstdefinestyle{mystyle}{
  backgroundcolor=\color{backcolour},   
  commentstyle=\color{red},
  keywordstyle=\color{blue},
  numberstyle=\tiny\color{codegray},
  stringstyle=\color{codegreen},
  basicstyle=\ttfamily,
  breakatwhitespace=false,         
  breaklines=true,                 
  captionpos=b,                    
  keepspaces=true,                 
  numbers=left,                    
  numbersep=5pt,                  
  showspaces=false,                
  showstringspaces=false,
  showtabs=false,                  
  tabsize=2
}

%"mystyle" code listing set
\lstset{style=mystyle}

\title{No Title}
\author{Claudeon R Susanto}
\date{}
\usepackage[T1]{fontenc}
\usepackage[utf8]{inputenc}
\usepackage[english]{babel}
\usepackage{lmodern}

\renewcommand{\familydefault}{\sfdefault}   % Supprime le serif (dyslexie)
\usepackage[font=sf, labelfont={sf}]{caption}
\usepackage{multicol}
\usepackage{makecell}
\renewcommand\theadalign{bc}
\renewcommand\theadfont{\bfseries}
\renewcommand\theadgape{\Gape[4pt]}
\renewcommand\cellgape{\Gape[4pt]}



% own commands
\newcommand{\eg}[0]{\textit{e.g. }}
\newcommand{\ie}[0]{\textit{i.e. }}
\newcommand{\impt}[0]{\textcolor{red}{\textbf{[IMPT] }}}


\renewcommand\thesubsection{\thesection.\arabic{subsection}}
\setlength{\columnseprule}{1pt}
\begin{document}
%\maketitle
\fontfamily{lmss}\selectfont
\begin{multicols}{2}
\section{Chapter 1: Accounting in Action}
\begin{description}
	\item[in the aggregate] accumulate transactions of the same type over a certain period and report the data as one amount in the company's financial statements
	\item[accounting] the entire process of identifying, recording, and communicating economic events (bookkeeping is part of recording only)\\
\end{description}
\textbf{Who uses accounting data?}
\begin{itemize}[topsep=0pt]
	\item \textbf{Internal users}\\
	\underline{Managerial accounting} provides internal reports to help users make decisions about their companies
	\begin{itemize}
		\item Management
		\item Employees
	\end{itemize}
	\item \textbf{External users} (investors and creditors, etc.)\\
	\underline{Financial accounting} provides economic and financial information for investors, creditors, and other external users
	\begin{itemize}
		\item Lenders
		\item Investors
		\item Competitors
		\item Government agencies\\
		IRS:
		SEC:
		\item The press
	\end{itemize}
\end{itemize}
\textbf{Measurement principles} (used by IFRS)
\begin{itemize}
	\item \impt Follow trade-offs between \textbf{relevance} (makes a difference in decision making) and \textbf{faithful representation} (factual and accurate)
	\item \impt Enhancing qualitative characteristics (\textbf{Comparability, Verifiability, Timeliness, Understandability})
	\item \textbf{Historical cost} principle: record assets at their initial cost when it was purchased
	\item \textbf{Fair value} principle: assets and liabilities should be reported at fair value (\underline{price received to sell an asset or settle a liability})
	\begin{itemize}
		\item Only used when asses are actively traded, otherwise rarely used
		\item Also used when market value info is available for certain assets\\
	\end{itemize}
\end{itemize}
\textbf{Accounting assumptions}
\begin{itemize}
	\item \textbf{Monetary unit} assumption: include only data that can be expressed in money terms
	\item \textbf{Economic entity} assumption: activities of the entity are separate and distinct from the activities of its owner and all other economic entities
	\begin{itemize}
		\item Proprietorship
		\begin{itemize}
			\item owned by \textbf{one} person
			\item the owner receives any profits and suffers any losses
			\item the owner has \textbf{unlimited liability} (liable for all debts of business)
			\item \textbf{no legal distinction} between the business as an economic unit and the owner
			\item Accounting records of the business activities are kept \textbf{separate} from owner's personal records
		\end{itemize}
		\item Partnership
		\begin{itemize}
			\item owned by \textbf{two or more} persons associated as partners
			\item each owner has \textbf{unlimited personal liability}
			\item for accounting purposes, partnership transactions are kept \textbf{separate} from personal activities
		\end{itemize}
		\item Corporation
		\begin{itemize}
			\item \textbf{separate legal identity} under corporation law
			\item ownership is divided into \textbf{transferable shares}: shareholders may transfer part or all of their ownership shares to other investors at any time
			\item holders of shares enjoy \textbf{limited liability}
			\item \textbf{Unlimited life}; ownership can be transferred without dissolving the corporation
		\end{itemize}
	\end{itemize}
\end{itemize}

\subsection{The Basic Accounting Equation}
$$\text{Assets} = \text{Liabilities} + \text{Equity}$$
\textbf{Assets}:  A resource controlled by the entity in the \textit{present} due to \textit{past} event that will give rise to \textit{future} benefits
	\begin{itemize}
		\item \underline{Cash}
		\item \underline{Accounts Receivable}
		\item \underline{Supplies}
		\item \underline{Equipment}
	\end{itemize}
\textbf{Liabilities}: A \textit{present} obligation arising from \textit{past} event that is expected to lead to a \textit{future} outflow of resources upon settlement
	 \begin{itemize}
	 	\item \underline{accounts payable}: purchase commodities/equipment on credit from suppliers
	 	\item \underline{note payable}: money borrowed
	 	\item \underline{salaries/wages payable}
	 	\item \underline{sales and real estate taxes payable}
	 	\item \impt Example: claim from an employee due to workplace accident which is highly likely to be settled in the future 
	 \end{itemize}
\textbf{Equity}: the ownership claim on a company (residual equity after creditors' claims are satisfied)
	 \begin{itemize}
	 	\item \textbf{Share capital-ordinary}: paid in by shareholders in exchange for the ordinary shares they purchase 
	 	\item \textbf{Retained earnings}
	 	\begin{itemize}
	 		\item \underline{Revenues}
	 		\item \underline{Expenses}
	 		\item \underline{Dividends}: increase in net assets, available to distribute to shareholders
	 	\end{itemize}
	 \end{itemize}

\subsection{Financial Statements}
\begin{enumerate}
	\item \textbf{Income statement (IS)} presents the revenues and expenses and resulting net  income or net loss for a specifi c period of time.
	\item \textbf{Retained earnings statement} summarizes the changes in retained earnings for a specific period of time.
	\item \textbf{Statement of financial position (SFP)} (sometimes referred to as a balance sheet) reports the assets, liabilities, and equity of a company at a specific date.
	\begin{itemize}
		\item Current and noncurrent assets/liabilities (can be turned into cash/settled within 1 year?)
		\item Preferably sorted from higher liquidity to lower 
		\item \impt Assets recorded at \textbf{cost/book value}, not market value
		\item \impt Revenue vs Loss/Gain\\
		Revenue: 
	\end{itemize}
	\item \textbf{Statement of cash flows (SCF)} summarizes information about the cash inflows (receipts) and outflows (payments) for a specific period of time.
	\item \textbf{Statement of comprehensive income (SCI)} presents other comprehensive income items that are not included in the determination of net income
\end{enumerate}
\end{multicols}






\end{document}